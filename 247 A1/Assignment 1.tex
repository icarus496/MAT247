\documentclass{article}
\usepackage[left=2cm,right=2cm,top=3cm,bottom=3cm]{geometry}
\usepackage{mathtools}
\usepackage{amsmath}
\usepackage{bbold}
\begin{document}
\subsection*{Q2}
	\begin{align*}
		f((0, 1, 0), (0, 1, 0)) &= 0 \cdot 0 + 0\cdot 0\\
		&=0
	\end{align*}
	However, $(0, 1, 0) \neq 0$, so $f$ does not satisfy the property of definiteness.
	
\subsection*{Q3}
Let $S$ be the set of inner products with the positivity condition. 
Let  $S^\prime$ be the set of inner products with the new condition. To start, let $f \in S$. Then, $f(v, v) > 0 \text{ } \forall v \in V \text{ with } v \neq 0$, so the new condition is satisfied since $\exists v \in V \text{ with } f(v, v) > 0$. So, $f \in S^\prime$, so $S \subseteq S^\prime$. 
Now consider the other direction: take a function $g \in S^\prime$. We know that by the new inner product definition, $\exists v \in V$ such that $g(v, v) > 0$. This $v$ must be nonzero, as we still have the definiteness axiom. Now, by contradiction, consider a nonzero vector $k$ orthogonal to $v$. Then, 
\begin{align*}
	\langle k , k \rangle &< 0\\
	\exists a: a \langle k, k \rangle &= - \langle v, v \rangle\\
	\langle \sqrt{a}k, \sqrt{a}k\rangle + \langle v, v \rangle &= 0\\
\end{align*}
So, 
\begin{equation*}
	\langle \sqrt{a}k + v, \sqrt{a}k + v \rangle = \langle \sqrt{a}k, \sqrt{a}k\rangle + \langle v, v \rangle = 0\\
\end{equation*}
Thus $\sqrt{a}k + v = 0 \Rightarrow k = -\frac{v}{\sqrt{a}}$, contradicting $k$ being orthogonal to $v$. So, $g(k, k) = 0$ for all $k$ orthogonal to $v$. Then, $\forall w \in V$, we can write $w$ as a sum of a scalar multiple of c plus a vector $k$ orthogonal to $v$. 
\begin{align*}
w&= cv + k\\
\langle w, w \rangle &= \langle cv+k, cv+k \rangle\\
&= \langle cv, cv \rangle + \langle k, k \rangle\\
\end{align*}
And since both terms of this sum are positive, we know that the inner product of w with itself is positive, so $g(w, w) > 0$ $\forall$ nonzero $w \in V$. That means $g \in S$, so  $S^\prime \subseteq S$, completing the proof.
\subsection*{Q8}
Let $u, v \in V$, $||u|| = ||v|| = 1$. Then, $|\rangle u, v \langle| = |1| = 1 = ||u|| ||v||$. So, by Cauchy-Schwartz, $u$ and $v$ are collinear, and have the same magnitude, so $u = -v or u = v$. However, if $u = -v$, $\langle u, v \rangle = \langle -v, v \rangle = - \langle v, v \rangle < 0$. But $\langle u, v \rangle = 1$, which is a contradiction, so $u = v$.

\subsection*{Q11}
Let $a,b \in \mathbb{R}^4$ with the inner product defined as the dot product. 
\begin{align*}
	a &= (\sqrt{a}, \sqrt{b}, \sqrt{c}, \sqrt{d})\\
	b &= (\frac{1}{\sqrt{a}}, \frac{1}{\sqrt{b}}, \frac{1}{\sqrt{c}},\frac{1}{\sqrt{d}})\\
	\text{then } |\langle a, b \rangle| &= |4| = 4\\ 
	||a|| &= \sqrt{\langle a, a \rangle} = \sqrt{a + b + c + d}\\
	||b|| &= \sqrt{\langle b, b \rangle} = \sqrt{\frac{1}{a} + \frac{1}{b} + \frac{1}{c} + \frac{1}{d}}\\
\end{align*}
So by Cauchy - Schwartz:
\begin{align*}
	|\langle a, b \rangle| &\leq ||a|| ||b||\\
	4 &\leq \sqrt{a + b + c + d} \sqrt{\frac{1}{a} + \frac{1}{b} + \frac{1}{c} + \frac{1}{d}}\\
	16 &\leq (a+b+c+d)(\frac{1}{a} + \frac{1}{b}+ \frac{1}{c}+\frac{1}{d})\\
\end{align*}
\subsection*{Q19}
\begin{align*}
	\langle u, v \rangle &= \frac{||u+v^2|| - ||u-v||^2}{4}\\
	&= \frac{\langle u+v, u+v \rangle - \langle u-v, u-v \rangle}{4}\\
	&= \frac{\langle u, u \rangle + \langle u , v \rangle + \langle v, u \rangle + \langle v, v \rangle - \langle u, u \rangle + \langle v, u \rangle + \langle u, v \rangle - \langle v, v \rangle}{4}\\
	&= \frac{4\langle u , v \rangle}{4}\\\
	&= \langle u, v \rangle
\end{align*}

\end{document}
