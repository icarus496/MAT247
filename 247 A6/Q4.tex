\documentclass{article}
\usepackage[left=2cm,right=2cm,top=3cm,bottom=3cm]{geometry}
\usepackage{mathtools}
\usepackage{amsmath}
\usepackage{bbold}
\begin{document}
\subsubsection*{Lemma 1}
Suppose you have a diagonal operator of the form $D =
\begin{pmatrix}
\lambda & 0 & ... & 0\\
0 & \lambda & ... & 0\\
... &... &... & ...\\
0 & ... & ... & \lambda\\
\end{pmatrix}
$ and some nilpotent matrix $N$ with all entries above the diagonal. Then, $DN = ND$ .
\subsubsection*{Proof:}
$D = \lambda I$, so $DN = \lambda IN = N\lambda I = ND$.
\subsubsection*{Lemma 2}
Suppose you have two $m \times m$ block diagonal operators:
\[
A = \begin{pmatrix} 
A_{1} & 0 ... & 0\\
0 & A_{2} & ... & 0\\
... &... &... & ...\\
0 & ... & ... & A_{m}\\
\end{pmatrix}
B = \begin{pmatrix}
B_{1} & 0 ... & 0\\
0 & B_{2} & ... & 0\\
... &... &... & ...\\
0 & ... & ... &B_{m}\\
\end{pmatrix}
\]
Where each $A_{i}, B_{i}$ is an $a_{i} \times a_{i}$ block. Then, we get:
\[
AB = \begin{pmatrix}
A_{1}B_{1} & 0 ... & 0\\
0 & A_{2}B_{2} & ... & 0\\
... &... &... & ...\\
0 & ... & ... & A_{m}B_{m}\\
\end{pmatrix}
\]
That is, the diagonal blocks of the product is the matrix product of the diagonal blocks.
\subsubsection*{Proof:}
For a general block diagonal matrix with $m$ $a_{i} \times a_{i}$ diagonal blocks $T_{1} ... T_{m}$, we can write:
\[T_{ij} = \begin{cases}
\sum_{p=1}^{k-1}a_{p} < i \leq \sum_{p=1}^{k}a_{p} \text{ and } \sum_{p=1}^{k-1}a_{p} < j \leq \sum_{p=1}^{k}a_{p}: & (T_{k})_{(i - \sum_{p=1}^{k-1}a_{p}, j - \sum_{p=1}^{k-1}a_{p})}\\
\text{ otherwise }: & 0
\end{cases}
\]
For simplicity, if $\sum_{p=1}^{k-1}a_{p} < i \leq \sum_{p=1}^{k}a_{p}$, define $i_{k}=i - \sum_{p=1}^{k-1}a_{p}$ Also define, and so $ 0 < i_{k} \leq a_{k}$. Similarly, define $j_{k} = j - \sum_{p=1}^{k-1}a_{p}$, so $ 0 < j_{k} \leq a_{k}$.
So, now we have:
\[AB_{i,j} = \sum_{q=0}^{n} A_{i,q}B_{q, j}
\]
For the first factor to be nonzero, with $0 < i_{k} \leq a_{k}$, we need $0 < q_{k} \leq a_{k}$. Given this, for the second factor to be nonzero, we need $0< j_{k} \leq a_{k}$. What this means is that everything term is 0 except the parts of the sum that satisfy the above conditions, so we get:
\[
AB_{i, j} = \sum_{q_k = 0}^{n-\sum_{p=1}^{k-1}a_{p}} A_{i_{k},q_{k}}B_{q_{k}, j_{k}} = (A_{k}B_{k})_{i_{k},j_{k}}
\]
As desired. 
\subsubsection*{The Actual Goddamn Proof}
Okay, luckily this part is pretty easy. Just put $T$ in block diagonal form, so 
\[
T = \begin{pmatrix} 
T_{1} & 0 ... & 0& 0\\
0 & T_{2} & ... & 0\\
... &... &... & ...\\
0 & ... & ... & T_{m}\\
\end{pmatrix}
\]
Let diag$(T_{i})$ equal the diagonal of one of the $T_{i}$ blocks. Then, define 
\[ D = \begin{pmatrix}
\text{diag}(T_{1}) & 0 ... & 0& 0\\
0 & \text{diag}(T_{2}) & ... & 0\\
... &... &... & ...\\
0 & ... & ... & \text{diag}(T_{m})\\
\end{pmatrix}
\]
Which is obviously diagonal, and $N$ as $T-D$. Then, $N$ is nilpotent, since we just took the diagonal away from an upper triangular matrix, meaning it is now strictly upper triangular. 
\[
N = \begin{pmatrix}
T_{1} - \text{diag}(T_{1}) & 0 ... & 0& 0\\
0 & T_{2} - \text{diag}(T_{2}) & ... & 0\\
... &... &... & ...\\
0 & ... & ... & T_{m}-\text{diag}(T_{m})\\
\end{pmatrix}
\]
Now, each $T_{i} - \text{diag}(T_{i}$ is nilpotent. So, we know from Lemma 2 that $DN$ can be found by multiplying each matrix on the diagonal of those operators together, and since the matrices on the diagonal of $D$ are diagonal (with only one value on the diagonal) and the matrices on the diagonal of $N$ are nilpotent, Lemma 1 shows each diag$(T_{i})$ commutes with each $T_{i} - \text{diag}(T_{i})$. Since the diagonal blocks commute, and $ND$ is just the product of their diagonal blocks, $ND = DN$. 
\end{document}
