\documentclass{article}
\usepackage[left=2cm,right=2cm,top=3cm,bottom=3cm]{geometry}
\usepackage{mathtools}
\usepackage{amsmath}
\usepackage{bbold}
\begin{document}
Let dim $v = n$. Then, suppose $v \in G(\lambda, T) = \text{null }(T-\lambda I)^{n}$, so $(T-\lambda I)^{n} v = 0$:
\begin{align*}
(T-\lambda I)^{n} &= \sum_{k=0}^{n} {n \choose k}(T)^{n-k}(\lambda)^{k}\\
\end{align*}
So:
\begin{align*}
&\sum_{k=0}^{n} {n \choose k}(T)^{n-k}(\lambda)^{k} v = 0\\
&\iff \lambda^{-n} T^{-n} \sum_{k=0}^{n} {n \choose k}(T)^{n-k}(\lambda)^{k} v = (\lambda^{-n} T^{-n}) 0\\
&\iff \sum_{k=0}^{n} {n \choose k} T^{-n} (T)^{n-k} \lambda^{-n}(\lambda)^{k} v = 0\\
&\iff (\sum_{k=0}^{n} {n \choose k} T^{-k}(\lambda)^{k-n}) v = 0\\
&\iff (\sum_{k=0}^{n} {n \choose k} T^{-k}(\frac{1}{\lambda})^{n-k}) v = 0\\
&\iff (T-\frac{1}{\lambda}I)^{n} v= 0
\end{align*}
Which implies that $v\in G(\frac{1}{\lambda}, T^{-1}) \iff v\in G(\lambda, T)$, as desired.
\end{document}
