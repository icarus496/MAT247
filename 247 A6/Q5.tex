\documentclass{article}
\usepackage[left=2cm,right=2cm,top=3cm,bottom=3cm]{geometry}
\usepackage{mathtools}
\usepackage{amsmath}
\usepackage{bbold}
\begin{document}
\subsubsection{Q5}
If $ND = DN$, then in general, there's a basis where $D$ is diagonal and $N$ is in block matrix form, because if there wasn't, they don't commute in general. Then, $N+D$ is also in block matrix form. However, $T = N + D$, and block-matrix form for $T$ is unique. $N$ in block matrix form must have 0s on the diagonal, and $D$ must have 0s everywhere else, so the block matrix form of $D$ and $N$ must be uniquely determined by $T$, as desired. 
\end{document}
